\documentclass{article}

% Language setting
% Replace `english' with e.g. `spanish' to change the document language
\usepackage[italian]{babel}

% Set page size and margins
% Replace `letterpaper' with `a4paper' for UK/EU standard size
\usepackage[letterpaper,top=2cm,bottom=2cm,left=3cm,right=3cm,marginparwidth=1.75cm]{geometry}

% Useful packages
\usepackage{amsmath}
\usepackage{amssymb}
\usepackage{graphicx}
\usepackage[colorlinks=true, allcolors=blue]{hyperref}
\usepackage{tikz}
\usepackage{tkz-euclide}
\usepackage{algorithm}%
\usepackage{algorithmicx}%
\usepackage{algpseudocode}%
\usepackage{listings}%
\usepackage{enumitem}
\usepackage{scrextend}
\usepackage{mathtools}

\title{K-Nearest Neighbors}
\author{Lorenzo Arcioni}

\begin{document}
\newtheorem{example}{Example}
\newtheorem{definition}{Definition}
\newtheorem{theorem}{Theorem}
\newtheorem{corollary}{Corollary}
\newtheorem{lemma}{Lemma}
\newtheorem{obs}{Observation}
\newtheorem{claim}{Claim}
\newtheorem{proposition}{Proposition}
\newtheorem{problem}{Problem}

\definecolor{def_color}{RGB}{237, 237, 237}
\definecolor{the_color}{RGB}{89, 79, 57}
\definecolor{pro_color_front}{RGB}{71, 93, 107}
\definecolor{pro_color_back}{RGB}{204, 206, 207}
\definecolor{lem_color}{RGB}{204, 206, 207}

\newcommand{\verteq}{\rotatebox{90}{$\,=$}}
\newcommand{\equalto}[2]{\underset{\scriptstyle\overset{\mkern4mu\verteq}{#2}}{#1}}
% Definizione del comando \argsort
\newcommand{\sort}{\operatorname*{sort}}

\maketitle

\begin{abstract}
    In questo articolo, presentiamo un'analisi approfondita dell'algoritmo K-Nearest Neighbors (KNN), 
    esaminandolo sia dal punto di vista teorico che pratico. L'algoritmo KNN è un metodo di 
    apprendimento supervisionato utilizzato per la classificazione e la regressione, basato sul 
    principio che oggetti simili sono vicini nello spazio delle caratteristiche. Iniziamo con una 
    descrizione dettagliata dei fondamenti teorici del KNN, compresa la definizione formale, 
    i criteri di scelta del parametro K e le metriche di distanza utilizzate per determinare la 
    vicinanza tra i dati. Successivamente, esploriamo le sue proprietà matematiche e discutiamo l'impatto 
    della dimensionalità dei dati e del rumore sulla sua performance. Attraverso un'analisi empirica, 
    confrontiamo l'efficacia del KNN con altri algoritmi di machine learning, utilizzando dataset 
    standard. Infine, esaminiamo le tecniche di ottimizzazione e miglioramento del KNN, come 
    la normalizzazione dei dati e l'uso di pesi nei vicini, per aumentare la precisione e l'efficienza 
    computazionale. Questo studio offre una visione completa del KNN, evidenziando i suoi punti di forza, 
    le sue limitazioni e le situazioni in cui è più adatto. 
\end{abstract}

\tableofcontents
\documentclass{article}

% Language setting
% Replace `english' with e.g. `spanish' to change the document language
\usepackage[italian]{babel}

% Set page size and margins
% Replace `letterpaper' with `a4paper' for UK/EU standard size
\usepackage[letterpaper,top=2cm,bottom=2cm,left=3cm,right=3cm,marginparwidth=1.75cm]{geometry}

% Useful packages
\usepackage{amsmath}
\usepackage{amssymb}
\usepackage{graphicx}
\usepackage[colorlinks=true, allcolors=blue]{hyperref}
\usepackage{tikz}
\usepackage{algorithm}%
\usepackage{algorithmicx}%
\usepackage{algpseudocode}%
\usepackage{listings}%
\usepackage{enumitem}
\usepackage{scrextend}
\usepackage{mathtools}

\title{K-Nearest Neighbors}
\author{Lorenzo Arcioni}

\begin{document}
\newtheorem{example}{Example}
\newtheorem{definition}{Definition}
\newtheorem{theorem}{Theorem}
\newtheorem{corollary}{Corollary}
\newtheorem{lemma}{Lemma}
\newtheorem{obs}{Observation}
\newtheorem{claim}{Claim}
\newtheorem{proposition}{Proposition}
\newtheorem{problem}{Problem}

\definecolor{def_color}{RGB}{237, 237, 237}
\definecolor{the_color}{RGB}{89, 79, 57}
\definecolor{pro_color_front}{RGB}{71, 93, 107}
\definecolor{pro_color_back}{RGB}{204, 206, 207}
\definecolor{lem_color}{RGB}{204, 206, 207}

\newcommand{\verteq}{\rotatebox{90}{$\,=$}}
\newcommand{\equalto}[2]{\underset{\scriptstyle\overset{\mkern4mu\verteq}{#2}}{#1}}
% Definizione del comando \argsort
\newcommand{\sort}{\operatorname*{sort}}

\maketitle

\begin{abstract}
    In questo articolo, presentiamo un'analisi approfondita dell'algoritmo K-Nearest Neighbors (KNN), 
    esaminandolo sia dal punto di vista teorico che pratico. L'algoritmo KNN è un metodo di 
    apprendimento supervisionato utilizzato per la classificazione e la regressione, basato sul 
    principio che oggetti simili sono vicini nello spazio delle caratteristiche. Iniziamo con una 
    descrizione dettagliata dei fondamenti teorici del KNN, compresa la definizione formale, 
    i criteri di scelta del parametro K e le metriche di distanza utilizzate per determinare la 
    vicinanza tra i dati. Successivamente, esploriamo le sue proprietà matematiche e discutiamo l'impatto 
    della dimensionalità dei dati e del rumore sulla sua performance. Attraverso un'analisi empirica, 
    confrontiamo l'efficacia del KNN con altri algoritmi di machine learning, utilizzando dataset 
    standard. Infine, esaminiamo le tecniche di ottimizzazione e miglioramento del KNN, come 
    la normalizzazione dei dati e l'uso di pesi nei vicini, per aumentare la precisione e l'efficienza 
    computazionale. Questo studio offre una visione completa del KNN, evidenziando i suoi punti di forza, 
    le sue limitazioni e le situazioni in cui è più adatto. 
\end{abstract}

\tableofcontents
\documentclass{article}

% Language setting
% Replace `english' with e.g. `spanish' to change the document language
\usepackage[italian]{babel}

% Set page size and margins
% Replace `letterpaper' with `a4paper' for UK/EU standard size
\usepackage[letterpaper,top=2cm,bottom=2cm,left=3cm,right=3cm,marginparwidth=1.75cm]{geometry}

% Useful packages
\usepackage{amsmath}
\usepackage{amssymb}
\usepackage{graphicx}
\usepackage[colorlinks=true, allcolors=blue]{hyperref}
\usepackage{tikz}
\usepackage{algorithm}%
\usepackage{algorithmicx}%
\usepackage{algpseudocode}%
\usepackage{listings}%
\usepackage{enumitem}
\usepackage{scrextend}
\usepackage{mathtools}

\title{K-Nearest Neighbors}
\author{Lorenzo Arcioni}

\begin{document}
\newtheorem{example}{Example}
\newtheorem{definition}{Definition}
\newtheorem{theorem}{Theorem}
\newtheorem{corollary}{Corollary}
\newtheorem{lemma}{Lemma}
\newtheorem{obs}{Observation}
\newtheorem{claim}{Claim}
\newtheorem{proposition}{Proposition}
\newtheorem{problem}{Problem}

\definecolor{def_color}{RGB}{237, 237, 237}
\definecolor{the_color}{RGB}{89, 79, 57}
\definecolor{pro_color_front}{RGB}{71, 93, 107}
\definecolor{pro_color_back}{RGB}{204, 206, 207}
\definecolor{lem_color}{RGB}{204, 206, 207}

\newcommand{\verteq}{\rotatebox{90}{$\,=$}}
\newcommand{\equalto}[2]{\underset{\scriptstyle\overset{\mkern4mu\verteq}{#2}}{#1}}
% Definizione del comando \argsort
\newcommand{\sort}{\operatorname*{sort}}

\maketitle

\begin{abstract}
    In questo articolo, presentiamo un'analisi approfondita dell'algoritmo K-Nearest Neighbors (KNN), 
    esaminandolo sia dal punto di vista teorico che pratico. L'algoritmo KNN è un metodo di 
    apprendimento supervisionato utilizzato per la classificazione e la regressione, basato sul 
    principio che oggetti simili sono vicini nello spazio delle caratteristiche. Iniziamo con una 
    descrizione dettagliata dei fondamenti teorici del KNN, compresa la definizione formale, 
    i criteri di scelta del parametro K e le metriche di distanza utilizzate per determinare la 
    vicinanza tra i dati. Successivamente, esploriamo le sue proprietà matematiche e discutiamo l'impatto 
    della dimensionalità dei dati e del rumore sulla sua performance. Attraverso un'analisi empirica, 
    confrontiamo l'efficacia del KNN con altri algoritmi di machine learning, utilizzando dataset 
    standard. Infine, esaminiamo le tecniche di ottimizzazione e miglioramento del KNN, come 
    la normalizzazione dei dati e l'uso di pesi nei vicini, per aumentare la precisione e l'efficienza 
    computazionale. Questo studio offre una visione completa del KNN, evidenziando i suoi punti di forza, 
    le sue limitazioni e le situazioni in cui è più adatto. 
\end{abstract}

\tableofcontents
\documentclass{article}

% Language setting
% Replace `english' with e.g. `spanish' to change the document language
\usepackage[italian]{babel}

% Set page size and margins
% Replace `letterpaper' with `a4paper' for UK/EU standard size
\usepackage[letterpaper,top=2cm,bottom=2cm,left=3cm,right=3cm,marginparwidth=1.75cm]{geometry}

% Useful packages
\usepackage{amsmath}
\usepackage{amssymb}
\usepackage{graphicx}
\usepackage[colorlinks=true, allcolors=blue]{hyperref}
\usepackage{tikz}
\usepackage{algorithm}%
\usepackage{algorithmicx}%
\usepackage{algpseudocode}%
\usepackage{listings}%
\usepackage{enumitem}
\usepackage{scrextend}
\usepackage{mathtools}

\title{K-Nearest Neighbors}
\author{Lorenzo Arcioni}

\begin{document}
\input{my_definitions}

\maketitle

\begin{abstract}
    In questo articolo, presentiamo un'analisi approfondita dell'algoritmo K-Nearest Neighbors (KNN), 
    esaminandolo sia dal punto di vista teorico che pratico. L'algoritmo KNN è un metodo di 
    apprendimento supervisionato utilizzato per la classificazione e la regressione, basato sul 
    principio che oggetti simili sono vicini nello spazio delle caratteristiche. Iniziamo con una 
    descrizione dettagliata dei fondamenti teorici del KNN, compresa la definizione formale, 
    i criteri di scelta del parametro K e le metriche di distanza utilizzate per determinare la 
    vicinanza tra i dati. Successivamente, esploriamo le sue proprietà matematiche e discutiamo l'impatto 
    della dimensionalità dei dati e del rumore sulla sua performance. Attraverso un'analisi empirica, 
    confrontiamo l'efficacia del KNN con altri algoritmi di machine learning, utilizzando dataset 
    standard. Infine, esaminiamo le tecniche di ottimizzazione e miglioramento del KNN, come 
    la normalizzazione dei dati e l'uso di pesi nei vicini, per aumentare la precisione e l'efficienza 
    computazionale. Questo studio offre una visione completa del KNN, evidenziando i suoi punti di forza, 
    le sue limitazioni e le situazioni in cui è più adatto. 
\end{abstract}

\tableofcontents
\input{main.toc}

\input{Chapters/01_Introduzione.tex}
\input{Chapters/02_Fondamenti_teorici.tex}
\input{Chapters/03_Analisi_Teorica.tex}

\nocite{*}
\bibliographystyle{unsrt}
\bibliography{sample.bib}

\end{document}


\section{Introduzione}

\subsection{Panoramica dell'algoritmo K-Nearest Neighbors (KNN)}
L'algoritmo K-Nearest Neighbors (KNN) è un metodo di apprendimento supervisionato utilizzato sia per problemi di classificazione che di regressione. La sua essenza risiede nel principio di vicinanza: gli oggetti simili tendono a trovarsi vicini nello spazio delle caratteristiche. Questa caratteristica rende il KNN intuitivo e semplice da implementare, pur essendo potente in molte applicazioni pratiche.

KNN è un metodo basato sulla prossimità, il che significa che, al momento della previsione per un nuovo dato, l'algoritmo cerca i K punti di addestramento più vicini (i "vicini") e utilizza le loro informazioni per fare la previsione. Per i problemi di classificazione, KNN assegna l'etichetta più comune tra i vicini; per i problemi di regressione, calcola la media dei valori dei vicini.

Un aspetto fondamentale del KNN è la scelta del parametro K, che rappresenta il numero di vicini da considerare. La scelta di K influisce significativamente sulla performance dell'algoritmo: un K troppo piccolo può rendere il modello sensibile al rumore (overfitting), mentre un K troppo grande può diluire la precisione del modello (underfitting).

Un altro elemento critico del KNN è la metrica di distanza utilizzata per determinare la vicinanza tra i punti. Le metriche di distanza più comuni includono la distanza euclidea, la distanza di Manhattan e la distanza di Minkowski, ognuna delle quali ha proprietà diverse che possono influenzare i risultati in base alla natura dei dati.

Nonostante la sua semplicità, KNN presenta alcune sfide, in particolare riguardo alla gestione di grandi dataset e alla sensibilità alla dimensionalità dei dati. Tuttavia, grazie alla sua natura non parametriche e alla facilità di implementazione, rimane un metodo popolare e ampiamente utilizzato in molte applicazioni di machine learning.

\subsection{Importanza e applicazioni del KNN}
\subsection{Obiettivi dell'articolo}
\section{Fondamenti Teorici del KNN}

\subsection{Definizione matematica formale}

Per formalizzare matematicamente l'algoritmo K-Nearest Neighbors (KNN), consideriamo un dataset di addestramento \( \mathcal{D} = \{(\mathbf{x}_i, y_i)\}_{i=1}^N \), dove \( \mathbf{x}_i \in \mathbb{R}^d \) rappresenta un punto dati con \( d \) caratteristiche e \( y_i \in \mathbb{R} \) (o \( y_i \in \{1, \ldots, C\} \) per la classificazione) rappresenta l'etichetta associata.

\subsubsection{Distanza tra punti dati}

Per determinare i \( K \) vicini più prossimi, è necessario definire una metrica di distanza \( d(\mathbf{x}, \mathbf{z}) \) tra due punti dati \( \mathbf{x} \) e \( \mathbf{z} \). Le metriche comunemente utilizzate includono:

\begin{itemize}
    \item \textbf{Distanza Euclidea:}
    \[
    d(\mathbf{x}, \mathbf{z}) = ||\mathbf{x} - \mathbf{z}|| = \sqrt{\sum_{j=1}^d (x_j - z_j)^2}
    \]

    \item \textbf{Distanza di Manhattan:}
    \[
    d(\mathbf{x}, \mathbf{z}) = \sum_{j=1}^d |x_j - z_j|
    \]

    \item \textbf{Distanza di Minkowski:}
    \[
    d(\mathbf{x}, \mathbf{z}) = \left( \sum_{j=1}^d |x_j - z_j|^p \right)^{\frac{1}{p}}
    \]
    dove \( p \) è un parametro positivo che determina la forma della distanza.
\end{itemize}

\subsubsection{Classificazione}

Nel contesto della classificazione, l'etichetta \( \hat{y} \) di un nuovo punto dati \( \mathbf{x} \) è determinata come segue:
\begin{enumerate}
    \item Calcolare la distanza tra \( \mathbf{x} \) e ogni punto dati \( \mathbf{x}_i \) nel dataset di addestramento.
    \item Identificare i \( K \) punti più vicini a \( \mathbf{x} \) utilizzando la metrica di distanza scelta.
    \item Assegnare a \( \mathbf{x} \) l'etichetta di classe più frequente tra i \( K \) vicini più prossimi. Formalmente,
    \[
    \hat{y} = \arg\max_{c \in \{1, \ldots, C\}} \sum_{i \in \mathcal{N}_K(\mathbf{x})} \mathbf{1}_{\{y_i = c\}}
    \]
    dove \( \mathcal{N}_K(\mathbf{x}) \) denota l'insieme dei \( K \) vicini più prossimi di \( \mathbf{x} \) e \( \mathbf{1}_{\{y_i = c\}} \) è una funzione indicatrice che vale 1 se \( y_i = c \) e 0 altrimenti.
\end{enumerate}

\subsubsection{Regressione}

Per la regressione, il valore predetto \( \hat{y} \) per un nuovo punto dati \( \mathbf{x} \) è calcolato come la media dei valori dei \( K \) vicini più prossimi:
\[
\hat{y} = \frac{1}{K} \sum_{i \in \mathcal{N}_K(\mathbf{x})} y_i
\]

Questa definizione matematica formale fornisce una chiara comprensione del funzionamento di base dell'algoritmo KNN, sia per la classificazione che per la regressione.

\subsection{Scelta del parametro K}
\subsection{Metriche di distanza}
\subsubsection{Distanza Euclidea}
\subsubsection{Distanza di Manhattan}
\subsubsection{Distanza di Minkowski}
\subsubsection{Altre metriche di distanza}

\section{Proprietà Matematiche e Analisi Teorica}
\subsection{La maledizione della dimensionalità}
\subsection{Complessità computazionale}
\subsection{Trade-off bias-varianza nel KNN}
\subsection{Interpretazione probabilistica del KNN}
\subsection{Comportamento asintotico e convergenza}
\section{Analisi Teorica}
\subsection{La maledizione della dimensionalità}
\subsection{Complessità computazionale}
\subsection{Trade-off bias-varianza nel KNN}
\subsection{Scelta di \( K \)}

Nel contesto del KNN, il (iper)parametro $K$ gioca un ruolo cruciale nel determinare il trade-off tra bias e varianza:

\subsubsection{Piccoli Valori di $K$}

Quando $K$ è piccolo (ad esempio, $K=1$), il modello tende a seguire 
molto da vicino i dati di addestramento. Questo può portare a un basso bias, 
poiché il modello è molto flessibile e può adattarsi alle particolarità dei dati 
di addestramento. Tuttavia, questo porta a una elevata varianza, poiché il modello 
è sensibile al rumore nei dati. In altre parole, un valore di $K$ troppo piccolo può 
causare overfitting.

\subsubsection{Grandi Valori di $K$}

Quando $K$ è grande (ad esempio, $K$ è una frazione significativa del dataset), 
il modello diventa più rigido. Esso effettua la media su un numero maggiore di punti, 
riducendo la varianza ma aumentando il bias. Questo significa che il modello potrebbe 
non catturare le complessità del dataset e potrebbe risultare in underfitting.

\subsubsection{Scegliere il Valore Ottimale di $K$}

La scelta ottimale di $K$ dipende dal dataset specifico. 
Una tecnica comune per trovare il valore ottimale di $K$ è utilizzare 
la validazione incrociata (cross-validation). In questa tecnica, 
il dataset viene diviso in $k$-folds (sottogruppi), e il modello 
viene addestrato e valutato $k$ volte, ogni volta utilizzando un 
diverso fold come set di validazione e il resto come set di addestramento. 
La media degli errori di validazione per ciascun valore di $K$ viene quindi 
utilizzata per selezionare il valore di $K$ che minimizza l'errore.

\subsection{Interpretazione probabilistica}

In teoria, per effettuare previsioni accurate, sarebbe ideale conoscere la distribuzione 
condizionale dei dati. Tuttavia, nella pratica, questa distribuzione è generalmente sconosciuta, 
rendendo impossibile una stima diretta basata su di essa. Nonostante ciò, metodi come il K-nearest 
neighbors (KNN) riescono comunque a fare previsioni accurate stimando tale distribuzione in maniera non parametrica.

Il KNN stima la distribuzione dei dati basandosi sui \( K \) punti di addestramento più vicini a un punto 
di test \( \hat{\mathbf{x}} \). La probabilità condizionale viene calcolata come la frazione dei punti 
in questo insieme che condividono la stessa caratteristica della variabile di interesse:

\[
Pr(Y = j \mid X = \mathbf{x}_0) = \frac{1}{K} \sum_{i \in N_0} I(y_i = j),
\]

dove \( N_0 \) rappresenta l'insieme dei \( K \) punti di addestramento più vicini a \( \mathbf{x}_0 \) e \( I(y_i = j) \) è una funzione indicatrice che vale 1 se \( y_i \) è uguale a \( j \) e 0 altrimenti.

Nonostante la semplicità del metodo, il KNN può spesso produrre previsioni molto efficaci, avvicinandosi al comportamento ottimale in molti scenari. Tuttavia, la scelta del parametro \( K \) è cruciale: un valore troppo piccolo di \( K \) rende il modello troppo flessibile e sensibile al rumore nei dati, mentre un valore troppo grande può rendere il modello eccessivamente rigido e incapace di catturare la struttura sottostante dei dati.

La relazione tra il tasso di errore di addestramento e quello di test non è sempre diretta. Aumentando la flessibilità del modello (diminuendo \( K \)), il tasso di errore di addestramento tende a diminuire, ma l'errore di test può aumentare se il modello soffre di overfitting. Questo comportamento è ben rappresentato dalla forma a U del grafico dell'errore di test in funzione di \( 1/K \).

La scelta del giusto livello di flessibilità è fondamentale per il successo di qualsiasi metodo di apprendimento statistico. Nel Capitolo 5, torneremo su questo argomento e discuteremo vari metodi per stimare i tassi di errore di test, al fine di scegliere il livello ottimale di flessibilità per un determinato metodo di apprendimento statistico.

\subsection{Comportamento asintotico e convergenza}


\nocite{*}
\bibliographystyle{unsrt}
\bibliography{sample.bib}

\end{document}


\section{Introduzione}

\subsection{Panoramica dell'algoritmo K-Nearest Neighbors (KNN)}
L'algoritmo K-Nearest Neighbors (KNN) è un metodo di apprendimento supervisionato utilizzato sia per problemi di classificazione che di regressione. La sua essenza risiede nel principio di vicinanza: gli oggetti simili tendono a trovarsi vicini nello spazio delle caratteristiche. Questa caratteristica rende il KNN intuitivo e semplice da implementare, pur essendo potente in molte applicazioni pratiche.

KNN è un metodo basato sulla prossimità, il che significa che, al momento della previsione per un nuovo dato, l'algoritmo cerca i K punti di addestramento più vicini (i "vicini") e utilizza le loro informazioni per fare la previsione. Per i problemi di classificazione, KNN assegna l'etichetta più comune tra i vicini; per i problemi di regressione, calcola la media dei valori dei vicini.

Un aspetto fondamentale del KNN è la scelta del parametro K, che rappresenta il numero di vicini da considerare. La scelta di K influisce significativamente sulla performance dell'algoritmo: un K troppo piccolo può rendere il modello sensibile al rumore (overfitting), mentre un K troppo grande può diluire la precisione del modello (underfitting).

Un altro elemento critico del KNN è la metrica di distanza utilizzata per determinare la vicinanza tra i punti. Le metriche di distanza più comuni includono la distanza euclidea, la distanza di Manhattan e la distanza di Minkowski, ognuna delle quali ha proprietà diverse che possono influenzare i risultati in base alla natura dei dati.

Nonostante la sua semplicità, KNN presenta alcune sfide, in particolare riguardo alla gestione di grandi dataset e alla sensibilità alla dimensionalità dei dati. Tuttavia, grazie alla sua natura non parametriche e alla facilità di implementazione, rimane un metodo popolare e ampiamente utilizzato in molte applicazioni di machine learning.

\subsection{Importanza e applicazioni del KNN}
\subsection{Obiettivi dell'articolo}
\section{Fondamenti Teorici del KNN}

\subsection{Definizione matematica formale}

Per formalizzare matematicamente l'algoritmo K-Nearest Neighbors (KNN), consideriamo un dataset di addestramento \( \mathcal{D} = \{(\mathbf{x}_i, y_i)\}_{i=1}^N \), dove \( \mathbf{x}_i \in \mathbb{R}^d \) rappresenta un punto dati con \( d \) caratteristiche e \( y_i \in \mathbb{R} \) (o \( y_i \in \{1, \ldots, C\} \) per la classificazione) rappresenta l'etichetta associata.

\subsubsection{Distanza tra punti dati}

Per determinare i \( K \) vicini più prossimi, è necessario definire una metrica di distanza \( d(\mathbf{x}, \mathbf{z}) \) tra due punti dati \( \mathbf{x} \) e \( \mathbf{z} \). Le metriche comunemente utilizzate includono:

\begin{itemize}
    \item \textbf{Distanza Euclidea:}
    \[
    d(\mathbf{x}, \mathbf{z}) = ||\mathbf{x} - \mathbf{z}|| = \sqrt{\sum_{j=1}^d (x_j - z_j)^2}
    \]

    \item \textbf{Distanza di Manhattan:}
    \[
    d(\mathbf{x}, \mathbf{z}) = \sum_{j=1}^d |x_j - z_j|
    \]

    \item \textbf{Distanza di Minkowski:}
    \[
    d(\mathbf{x}, \mathbf{z}) = \left( \sum_{j=1}^d |x_j - z_j|^p \right)^{\frac{1}{p}}
    \]
    dove \( p \) è un parametro positivo che determina la forma della distanza.
\end{itemize}

\subsubsection{Classificazione}

Nel contesto della classificazione, l'etichetta \( \hat{y} \) di un nuovo punto dati \( \mathbf{x} \) è determinata come segue:
\begin{enumerate}
    \item Calcolare la distanza tra \( \mathbf{x} \) e ogni punto dati \( \mathbf{x}_i \) nel dataset di addestramento.
    \item Identificare i \( K \) punti più vicini a \( \mathbf{x} \) utilizzando la metrica di distanza scelta.
    \item Assegnare a \( \mathbf{x} \) l'etichetta di classe più frequente tra i \( K \) vicini più prossimi. Formalmente,
    \[
    \hat{y} = \arg\max_{c \in \{1, \ldots, C\}} \sum_{i \in \mathcal{N}_K(\mathbf{x})} \mathbf{1}_{\{y_i = c\}}
    \]
    dove \( \mathcal{N}_K(\mathbf{x}) \) denota l'insieme dei \( K \) vicini più prossimi di \( \mathbf{x} \) e \( \mathbf{1}_{\{y_i = c\}} \) è una funzione indicatrice che vale 1 se \( y_i = c \) e 0 altrimenti.
\end{enumerate}

\subsubsection{Regressione}

Per la regressione, il valore predetto \( \hat{y} \) per un nuovo punto dati \( \mathbf{x} \) è calcolato come la media dei valori dei \( K \) vicini più prossimi:
\[
\hat{y} = \frac{1}{K} \sum_{i \in \mathcal{N}_K(\mathbf{x})} y_i
\]

Questa definizione matematica formale fornisce una chiara comprensione del funzionamento di base dell'algoritmo KNN, sia per la classificazione che per la regressione.

\subsection{Scelta del parametro K}
\subsection{Metriche di distanza}
\subsubsection{Distanza Euclidea}
\subsubsection{Distanza di Manhattan}
\subsubsection{Distanza di Minkowski}
\subsubsection{Altre metriche di distanza}

\section{Proprietà Matematiche e Analisi Teorica}
\subsection{La maledizione della dimensionalità}
\subsection{Complessità computazionale}
\subsection{Trade-off bias-varianza nel KNN}
\subsection{Interpretazione probabilistica del KNN}
\subsection{Comportamento asintotico e convergenza}
\section{Analisi Teorica}
\subsection{La maledizione della dimensionalità}
\subsection{Complessità computazionale}
\subsection{Trade-off bias-varianza nel KNN}
\subsection{Scelta di \( K \)}

Nel contesto del KNN, il (iper)parametro $K$ gioca un ruolo cruciale nel determinare il trade-off tra bias e varianza:

\subsubsection{Piccoli Valori di $K$}

Quando $K$ è piccolo (ad esempio, $K=1$), il modello tende a seguire 
molto da vicino i dati di addestramento. Questo può portare a un basso bias, 
poiché il modello è molto flessibile e può adattarsi alle particolarità dei dati 
di addestramento. Tuttavia, questo porta a una elevata varianza, poiché il modello 
è sensibile al rumore nei dati. In altre parole, un valore di $K$ troppo piccolo può 
causare overfitting.

\subsubsection{Grandi Valori di $K$}

Quando $K$ è grande (ad esempio, $K$ è una frazione significativa del dataset), 
il modello diventa più rigido. Esso effettua la media su un numero maggiore di punti, 
riducendo la varianza ma aumentando il bias. Questo significa che il modello potrebbe 
non catturare le complessità del dataset e potrebbe risultare in underfitting.

\subsubsection{Scegliere il Valore Ottimale di $K$}

La scelta ottimale di $K$ dipende dal dataset specifico. 
Una tecnica comune per trovare il valore ottimale di $K$ è utilizzare 
la validazione incrociata (cross-validation). In questa tecnica, 
il dataset viene diviso in $k$-folds (sottogruppi), e il modello 
viene addestrato e valutato $k$ volte, ogni volta utilizzando un 
diverso fold come set di validazione e il resto come set di addestramento. 
La media degli errori di validazione per ciascun valore di $K$ viene quindi 
utilizzata per selezionare il valore di $K$ che minimizza l'errore.

\subsection{Interpretazione probabilistica}

In teoria, per effettuare previsioni accurate, sarebbe ideale conoscere la distribuzione 
condizionale dei dati. Tuttavia, nella pratica, questa distribuzione è generalmente sconosciuta, 
rendendo impossibile una stima diretta basata su di essa. Nonostante ciò, metodi come il K-nearest 
neighbors (KNN) riescono comunque a fare previsioni accurate stimando tale distribuzione in maniera non parametrica.

Il KNN stima la distribuzione dei dati basandosi sui \( K \) punti di addestramento più vicini a un punto 
di test \( \hat{\mathbf{x}} \). La probabilità condizionale viene calcolata come la frazione dei punti 
in questo insieme che condividono la stessa caratteristica della variabile di interesse:

\[
Pr(Y = j \mid X = \mathbf{x}_0) = \frac{1}{K} \sum_{i \in N_0} I(y_i = j),
\]

dove \( N_0 \) rappresenta l'insieme dei \( K \) punti di addestramento più vicini a \( \mathbf{x}_0 \) e \( I(y_i = j) \) è una funzione indicatrice che vale 1 se \( y_i \) è uguale a \( j \) e 0 altrimenti.

Nonostante la semplicità del metodo, il KNN può spesso produrre previsioni molto efficaci, avvicinandosi al comportamento ottimale in molti scenari. Tuttavia, la scelta del parametro \( K \) è cruciale: un valore troppo piccolo di \( K \) rende il modello troppo flessibile e sensibile al rumore nei dati, mentre un valore troppo grande può rendere il modello eccessivamente rigido e incapace di catturare la struttura sottostante dei dati.

La relazione tra il tasso di errore di addestramento e quello di test non è sempre diretta. Aumentando la flessibilità del modello (diminuendo \( K \)), il tasso di errore di addestramento tende a diminuire, ma l'errore di test può aumentare se il modello soffre di overfitting. Questo comportamento è ben rappresentato dalla forma a U del grafico dell'errore di test in funzione di \( 1/K \).

La scelta del giusto livello di flessibilità è fondamentale per il successo di qualsiasi metodo di apprendimento statistico. Nel Capitolo 5, torneremo su questo argomento e discuteremo vari metodi per stimare i tassi di errore di test, al fine di scegliere il livello ottimale di flessibilità per un determinato metodo di apprendimento statistico.

\subsection{Comportamento asintotico e convergenza}


\nocite{*}
\bibliographystyle{unsrt}
\bibliography{sample.bib}

\end{document}


\section{Introduzione}

\subsection{Panoramica dell'algoritmo K-Nearest Neighbors (KNN)}
L'algoritmo K-Nearest Neighbors (KNN) è un metodo di apprendimento supervisionato utilizzato sia per problemi di classificazione che di regressione. La sua essenza risiede nel principio di vicinanza: gli oggetti simili tendono a trovarsi vicini nello spazio delle caratteristiche. Questa caratteristica rende il KNN intuitivo e semplice da implementare, pur essendo potente in molte applicazioni pratiche.

KNN è un metodo basato sulla prossimità, il che significa che, al momento della previsione per un nuovo dato, l'algoritmo cerca i K punti di addestramento più vicini (i "vicini") e utilizza le loro informazioni per fare la previsione. Per i problemi di classificazione, KNN assegna l'etichetta più comune tra i vicini; per i problemi di regressione, calcola la media dei valori dei vicini.

Un aspetto fondamentale del KNN è la scelta del parametro K, che rappresenta il numero di vicini da considerare. La scelta di K influisce significativamente sulla performance dell'algoritmo: un K troppo piccolo può rendere il modello sensibile al rumore (overfitting), mentre un K troppo grande può diluire la precisione del modello (underfitting).

Un altro elemento critico del KNN è la metrica di distanza utilizzata per determinare la vicinanza tra i punti. Le metriche di distanza più comuni includono la distanza euclidea, la distanza di Manhattan e la distanza di Minkowski, ognuna delle quali ha proprietà diverse che possono influenzare i risultati in base alla natura dei dati.

Nonostante la sua semplicità, KNN presenta alcune sfide, in particolare riguardo alla gestione di grandi dataset e alla sensibilità alla dimensionalità dei dati. Tuttavia, grazie alla sua natura non parametriche e alla facilità di implementazione, rimane un metodo popolare e ampiamente utilizzato in molte applicazioni di machine learning.

\subsection{Importanza e applicazioni del KNN}
\subsection{Obiettivi dell'articolo}
\section{Fondamenti Teorici del KNN}

\subsection{Definizione matematica formale}

Per formalizzare matematicamente l'algoritmo K-Nearest Neighbors (KNN), consideriamo un dataset di addestramento \( \mathcal{D} = \{(\mathbf{x}_i, y_i)\}_{i=1}^N \), dove \( \mathbf{x}_i \in \mathbb{R}^d \) rappresenta un punto dati con \( d \) caratteristiche e \( y_i \in \mathbb{R} \) (o \( y_i \in \{1, \ldots, C\} \) per la classificazione) rappresenta l'etichetta associata.

\subsubsection{Distanza tra punti dati}

Per determinare i \( K \) vicini più prossimi, è necessario definire una metrica di distanza \( d(\mathbf{x}, \mathbf{z}) \) tra due punti dati \( \mathbf{x} \) e \( \mathbf{z} \). Le metriche comunemente utilizzate includono:

\begin{itemize}
    \item \textbf{Distanza Euclidea:}
    \[
    d(\mathbf{x}, \mathbf{z}) = ||\mathbf{x} - \mathbf{z}|| = \sqrt{\sum_{j=1}^d (x_j - z_j)^2}
    \]

    \item \textbf{Distanza di Manhattan:}
    \[
    d(\mathbf{x}, \mathbf{z}) = \sum_{j=1}^d |x_j - z_j|
    \]

    \item \textbf{Distanza di Minkowski:}
    \[
    d(\mathbf{x}, \mathbf{z}) = \left( \sum_{j=1}^d |x_j - z_j|^p \right)^{\frac{1}{p}}
    \]
    dove \( p \) è un parametro positivo che determina la forma della distanza.
\end{itemize}

\subsubsection{Classificazione}

Nel contesto della classificazione, l'etichetta \( \hat{y} \) di un nuovo punto dati \( \mathbf{x} \) è determinata come segue:
\begin{enumerate}
    \item Calcolare la distanza tra \( \mathbf{x} \) e ogni punto dati \( \mathbf{x}_i \) nel dataset di addestramento.
    \item Identificare i \( K \) punti più vicini a \( \mathbf{x} \) utilizzando la metrica di distanza scelta.
    \item Assegnare a \( \mathbf{x} \) l'etichetta di classe più frequente tra i \( K \) vicini più prossimi. Formalmente,
    \[
    \hat{y} = \arg\max_{c \in \{1, \ldots, C\}} \sum_{i \in \mathcal{N}_K(\mathbf{x})} \mathbf{1}_{\{y_i = c\}}
    \]
    dove \( \mathcal{N}_K(\mathbf{x}) \) denota l'insieme dei \( K \) vicini più prossimi di \( \mathbf{x} \) e \( \mathbf{1}_{\{y_i = c\}} \) è una funzione indicatrice che vale 1 se \( y_i = c \) e 0 altrimenti.
\end{enumerate}

\subsubsection{Regressione}

Per la regressione, il valore predetto \( \hat{y} \) per un nuovo punto dati \( \mathbf{x} \) è calcolato come la media dei valori dei \( K \) vicini più prossimi:
\[
\hat{y} = \frac{1}{K} \sum_{i \in \mathcal{N}_K(\mathbf{x})} y_i
\]

Questa definizione matematica formale fornisce una chiara comprensione del funzionamento di base dell'algoritmo KNN, sia per la classificazione che per la regressione.

\subsection{Scelta del parametro K}
\subsection{Metriche di distanza}
\subsubsection{Distanza Euclidea}
\subsubsection{Distanza di Manhattan}
\subsubsection{Distanza di Minkowski}
\subsubsection{Altre metriche di distanza}

\section{Proprietà Matematiche e Analisi Teorica}
\subsection{La maledizione della dimensionalità}
\subsection{Complessità computazionale}
\subsection{Trade-off bias-varianza nel KNN}
\subsection{Interpretazione probabilistica del KNN}
\subsection{Comportamento asintotico e convergenza}
\section{Analisi Teorica}
\subsection{La maledizione della dimensionalità}
\subsection{Complessità computazionale}
\subsection{Trade-off bias-varianza nel KNN}
\subsection{Scelta di \( K \)}

Nel contesto del KNN, il (iper)parametro $K$ gioca un ruolo cruciale nel determinare il trade-off tra bias e varianza:

\subsubsection{Piccoli Valori di $K$}

Quando $K$ è piccolo (ad esempio, $K=1$), il modello tende a seguire 
molto da vicino i dati di addestramento. Questo può portare a un basso bias, 
poiché il modello è molto flessibile e può adattarsi alle particolarità dei dati 
di addestramento. Tuttavia, questo porta a una elevata varianza, poiché il modello 
è sensibile al rumore nei dati. In altre parole, un valore di $K$ troppo piccolo può 
causare overfitting.

\subsubsection{Grandi Valori di $K$}

Quando $K$ è grande (ad esempio, $K$ è una frazione significativa del dataset), 
il modello diventa più rigido. Esso effettua la media su un numero maggiore di punti, 
riducendo la varianza ma aumentando il bias. Questo significa che il modello potrebbe 
non catturare le complessità del dataset e potrebbe risultare in underfitting.

\subsubsection{Scegliere il Valore Ottimale di $K$}

La scelta ottimale di $K$ dipende dal dataset specifico. 
Una tecnica comune per trovare il valore ottimale di $K$ è utilizzare 
la validazione incrociata (cross-validation). In questa tecnica, 
il dataset viene diviso in $k$-folds (sottogruppi), e il modello 
viene addestrato e valutato $k$ volte, ogni volta utilizzando un 
diverso fold come set di validazione e il resto come set di addestramento. 
La media degli errori di validazione per ciascun valore di $K$ viene quindi 
utilizzata per selezionare il valore di $K$ che minimizza l'errore.

\subsection{Interpretazione probabilistica}

In teoria, per effettuare previsioni accurate, sarebbe ideale conoscere la distribuzione 
condizionale dei dati. Tuttavia, nella pratica, questa distribuzione è generalmente sconosciuta, 
rendendo impossibile una stima diretta basata su di essa. Nonostante ciò, metodi come il K-nearest 
neighbors (KNN) riescono comunque a fare previsioni accurate stimando tale distribuzione in maniera non parametrica.

Il KNN stima la distribuzione dei dati basandosi sui \( K \) punti di addestramento più vicini a un punto 
di test \( \hat{\mathbf{x}} \). La probabilità condizionale viene calcolata come la frazione dei punti 
in questo insieme che condividono la stessa caratteristica della variabile di interesse:

\[
Pr(Y = j \mid X = \mathbf{x}_0) = \frac{1}{K} \sum_{i \in N_0} I(y_i = j),
\]

dove \( N_0 \) rappresenta l'insieme dei \( K \) punti di addestramento più vicini a \( \mathbf{x}_0 \) e \( I(y_i = j) \) è una funzione indicatrice che vale 1 se \( y_i \) è uguale a \( j \) e 0 altrimenti.

Nonostante la semplicità del metodo, il KNN può spesso produrre previsioni molto efficaci, avvicinandosi al comportamento ottimale in molti scenari. Tuttavia, la scelta del parametro \( K \) è cruciale: un valore troppo piccolo di \( K \) rende il modello troppo flessibile e sensibile al rumore nei dati, mentre un valore troppo grande può rendere il modello eccessivamente rigido e incapace di catturare la struttura sottostante dei dati.

La relazione tra il tasso di errore di addestramento e quello di test non è sempre diretta. Aumentando la flessibilità del modello (diminuendo \( K \)), il tasso di errore di addestramento tende a diminuire, ma l'errore di test può aumentare se il modello soffre di overfitting. Questo comportamento è ben rappresentato dalla forma a U del grafico dell'errore di test in funzione di \( 1/K \).

La scelta del giusto livello di flessibilità è fondamentale per il successo di qualsiasi metodo di apprendimento statistico. Nel Capitolo 5, torneremo su questo argomento e discuteremo vari metodi per stimare i tassi di errore di test, al fine di scegliere il livello ottimale di flessibilità per un determinato metodo di apprendimento statistico.

\subsection{Comportamento asintotico e convergenza}

\section{Ottimizzazioni}
\label{sec:ottimizzazioni}

\subsection{KD-Tree}
\label{subsec:kd_tree}

\subsection{Ball Tree}
\label{subsec:ball_tree}

\subsection{Hashing}
\label{subsec:hashing}

\subsection{Algoritmi Approximate Nearest Neighbors (ANN)}
\label{subsec:approximate_nearest_neighbors}


\nocite{*}
\bibliographystyle{unsrt}
\bibliography{sample.bib}

\end{document}